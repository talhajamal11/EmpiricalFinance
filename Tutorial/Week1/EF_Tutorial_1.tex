% Options for packages loaded elsewhere
\PassOptionsToPackage{unicode}{hyperref}
\PassOptionsToPackage{hyphens}{url}
%
\documentclass[
]{article}
\usepackage{amsmath,amssymb}
\usepackage{iftex}
\ifPDFTeX
  \usepackage[T1]{fontenc}
  \usepackage[utf8]{inputenc}
  \usepackage{textcomp} % provide euro and other symbols
\else % if luatex or xetex
  \usepackage{unicode-math} % this also loads fontspec
  \defaultfontfeatures{Scale=MatchLowercase}
  \defaultfontfeatures[\rmfamily]{Ligatures=TeX,Scale=1}
\fi
\usepackage{lmodern}
\ifPDFTeX\else
  % xetex/luatex font selection
\fi
% Use upquote if available, for straight quotes in verbatim environments
\IfFileExists{upquote.sty}{\usepackage{upquote}}{}
\IfFileExists{microtype.sty}{% use microtype if available
  \usepackage[]{microtype}
  \UseMicrotypeSet[protrusion]{basicmath} % disable protrusion for tt fonts
}{}
\makeatletter
\@ifundefined{KOMAClassName}{% if non-KOMA class
  \IfFileExists{parskip.sty}{%
    \usepackage{parskip}
  }{% else
    \setlength{\parindent}{0pt}
    \setlength{\parskip}{6pt plus 2pt minus 1pt}}
}{% if KOMA class
  \KOMAoptions{parskip=half}}
\makeatother
\usepackage{xcolor}
\usepackage[margin=1in]{geometry}
\usepackage{color}
\usepackage{fancyvrb}
\newcommand{\VerbBar}{|}
\newcommand{\VERB}{\Verb[commandchars=\\\{\}]}
\DefineVerbatimEnvironment{Highlighting}{Verbatim}{commandchars=\\\{\}}
% Add ',fontsize=\small' for more characters per line
\usepackage{framed}
\definecolor{shadecolor}{RGB}{248,248,248}
\newenvironment{Shaded}{\begin{snugshade}}{\end{snugshade}}
\newcommand{\AlertTok}[1]{\textcolor[rgb]{0.94,0.16,0.16}{#1}}
\newcommand{\AnnotationTok}[1]{\textcolor[rgb]{0.56,0.35,0.01}{\textbf{\textit{#1}}}}
\newcommand{\AttributeTok}[1]{\textcolor[rgb]{0.13,0.29,0.53}{#1}}
\newcommand{\BaseNTok}[1]{\textcolor[rgb]{0.00,0.00,0.81}{#1}}
\newcommand{\BuiltInTok}[1]{#1}
\newcommand{\CharTok}[1]{\textcolor[rgb]{0.31,0.60,0.02}{#1}}
\newcommand{\CommentTok}[1]{\textcolor[rgb]{0.56,0.35,0.01}{\textit{#1}}}
\newcommand{\CommentVarTok}[1]{\textcolor[rgb]{0.56,0.35,0.01}{\textbf{\textit{#1}}}}
\newcommand{\ConstantTok}[1]{\textcolor[rgb]{0.56,0.35,0.01}{#1}}
\newcommand{\ControlFlowTok}[1]{\textcolor[rgb]{0.13,0.29,0.53}{\textbf{#1}}}
\newcommand{\DataTypeTok}[1]{\textcolor[rgb]{0.13,0.29,0.53}{#1}}
\newcommand{\DecValTok}[1]{\textcolor[rgb]{0.00,0.00,0.81}{#1}}
\newcommand{\DocumentationTok}[1]{\textcolor[rgb]{0.56,0.35,0.01}{\textbf{\textit{#1}}}}
\newcommand{\ErrorTok}[1]{\textcolor[rgb]{0.64,0.00,0.00}{\textbf{#1}}}
\newcommand{\ExtensionTok}[1]{#1}
\newcommand{\FloatTok}[1]{\textcolor[rgb]{0.00,0.00,0.81}{#1}}
\newcommand{\FunctionTok}[1]{\textcolor[rgb]{0.13,0.29,0.53}{\textbf{#1}}}
\newcommand{\ImportTok}[1]{#1}
\newcommand{\InformationTok}[1]{\textcolor[rgb]{0.56,0.35,0.01}{\textbf{\textit{#1}}}}
\newcommand{\KeywordTok}[1]{\textcolor[rgb]{0.13,0.29,0.53}{\textbf{#1}}}
\newcommand{\NormalTok}[1]{#1}
\newcommand{\OperatorTok}[1]{\textcolor[rgb]{0.81,0.36,0.00}{\textbf{#1}}}
\newcommand{\OtherTok}[1]{\textcolor[rgb]{0.56,0.35,0.01}{#1}}
\newcommand{\PreprocessorTok}[1]{\textcolor[rgb]{0.56,0.35,0.01}{\textit{#1}}}
\newcommand{\RegionMarkerTok}[1]{#1}
\newcommand{\SpecialCharTok}[1]{\textcolor[rgb]{0.81,0.36,0.00}{\textbf{#1}}}
\newcommand{\SpecialStringTok}[1]{\textcolor[rgb]{0.31,0.60,0.02}{#1}}
\newcommand{\StringTok}[1]{\textcolor[rgb]{0.31,0.60,0.02}{#1}}
\newcommand{\VariableTok}[1]{\textcolor[rgb]{0.00,0.00,0.00}{#1}}
\newcommand{\VerbatimStringTok}[1]{\textcolor[rgb]{0.31,0.60,0.02}{#1}}
\newcommand{\WarningTok}[1]{\textcolor[rgb]{0.56,0.35,0.01}{\textbf{\textit{#1}}}}
\usepackage{graphicx}
\makeatletter
\def\maxwidth{\ifdim\Gin@nat@width>\linewidth\linewidth\else\Gin@nat@width\fi}
\def\maxheight{\ifdim\Gin@nat@height>\textheight\textheight\else\Gin@nat@height\fi}
\makeatother
% Scale images if necessary, so that they will not overflow the page
% margins by default, and it is still possible to overwrite the defaults
% using explicit options in \includegraphics[width, height, ...]{}
\setkeys{Gin}{width=\maxwidth,height=\maxheight,keepaspectratio}
% Set default figure placement to htbp
\makeatletter
\def\fps@figure{htbp}
\makeatother
\setlength{\emergencystretch}{3em} % prevent overfull lines
\providecommand{\tightlist}{%
  \setlength{\itemsep}{0pt}\setlength{\parskip}{0pt}}
\setcounter{secnumdepth}{-\maxdimen} % remove section numbering
\ifLuaTeX
  \usepackage{selnolig}  % disable illegal ligatures
\fi
\IfFileExists{bookmark.sty}{\usepackage{bookmark}}{\usepackage{hyperref}}
\IfFileExists{xurl.sty}{\usepackage{xurl}}{} % add URL line breaks if available
\urlstyle{same}
\hypersetup{
  pdftitle={EF\_Tutorial\_1},
  pdfauthor={Gillian},
  hidelinks,
  pdfcreator={LaTeX via pandoc}}

\title{EF\_Tutorial\_1}
\author{Gillian}
\date{}

\begin{document}
\maketitle

\hypertarget{packages}{%
\subsection{Packages}\label{packages}}

\begin{Shaded}
\begin{Highlighting}[]
\NormalTok{package\_name }\OtherTok{\textless{}{-}} \StringTok{"tidyverse"}

\CommentTok{\# Check if the package is installed}
\ControlFlowTok{if}\NormalTok{ (}\SpecialCharTok{!}\FunctionTok{requireNamespace}\NormalTok{(package\_name, }\AttributeTok{quietly =} \ConstantTok{TRUE}\NormalTok{)) \{}
  \CommentTok{\# If not installed, install the package}
  \FunctionTok{install.packages}\NormalTok{(package\_name)\}}

\FunctionTok{library}\NormalTok{(package\_name, }\AttributeTok{character.only =} \ConstantTok{TRUE}\NormalTok{)}
\end{Highlighting}
\end{Shaded}

\#\#Basic functions If you are not familiar with the function and want
to get more info, add ``?'' at the front of the function to check its
documentation (e.g.~?read\_csv).

\hypertarget{data-importing}{%
\subsection{Data importing}\label{data-importing}}

\begin{Shaded}
\begin{Highlighting}[]
\FunctionTok{library}\NormalTok{(tidyverse)}
\NormalTok{mydata }\OtherTok{\textless{}{-}} \FunctionTok{read\_csv}\NormalTok{(}\StringTok{"eaef21.csv"}\NormalTok{) }\CommentTok{\# load the dataset eaef21.csv into R }
\CommentTok{\#mydata \textless{}{-} read\_csv("C:/Users/gilli/OneDrive/Documents/EF\_tutorial\_1/eaef21.csv")}

\CommentTok{\#Sometimes your data is only available as a xlxs file.}
\CommentTok{\#example \textless{}{-}read\_exel(example)}

\CommentTok{\#More info on importing various types of data https://github.com/rstudio/cheatsheets/blob/main/data{-}import.pdf}

\CommentTok{\#Here is the chapter on data importing from "R for Data Sicence" https://r4ds.had.co.nz/data{-}import.html}
\end{Highlighting}
\end{Shaded}

\#API \#Can use API packages to access data without downloading it. This
is very useful when dealing with large dataset or with ones that update
regularly

\begin{Shaded}
\begin{Highlighting}[]
\CommentTok{\#install.packages("wbstats")}
\FunctionTok{library}\NormalTok{(wbstats)}
\NormalTok{gdp\_data }\OtherTok{\textless{}{-}} \FunctionTok{wb\_data}\NormalTok{(}\AttributeTok{country =} \FunctionTok{c}\NormalTok{(}\StringTok{"AF"}\NormalTok{,}\StringTok{"CA"}\NormalTok{), }\AttributeTok{indicator =} \StringTok{"NY.GDP.PCAP.CD"}\NormalTok{)}
\end{Highlighting}
\end{Shaded}

\begin{Shaded}
\begin{Highlighting}[]
\FunctionTok{class}\NormalTok{(mydata)   }\CommentTok{\# class will show us the type of object \textquotesingle{}mydata\textquotesingle{}}
\end{Highlighting}
\end{Shaded}

\begin{verbatim}
## [1] "spec_tbl_df" "tbl_df"      "tbl"         "data.frame"
\end{verbatim}

\begin{Shaded}
\begin{Highlighting}[]
\NormalTok{mydata }\OtherTok{\textless{}{-}} \FunctionTok{as.data.frame}\NormalTok{(mydata)   }\CommentTok{\# change object to data.frame}
\FunctionTok{class}\NormalTok{(mydata)}
\end{Highlighting}
\end{Shaded}

\begin{verbatim}
## [1] "data.frame"
\end{verbatim}

\begin{Shaded}
\begin{Highlighting}[]
\FunctionTok{head}\NormalTok{(mydata)     }\CommentTok{\# shows the first 6 rows of your dataset}
\FunctionTok{tail}\NormalTok{(mydata)    }\CommentTok{\# shows the last 6 rows of your dataset}
\end{Highlighting}
\end{Shaded}

\begin{Shaded}
\begin{Highlighting}[]
\FunctionTok{summary}\NormalTok{(mydata) }\CommentTok{\# gives you a simple summary of your dataset (not very nice looking)}
\end{Highlighting}
\end{Shaded}

\begin{verbatim}
##        ID            FEMALE         MALE        ETHBLACK         ETHHISP       
##  Min.   :   18   Min.   :0.0   Min.   :0.0   Min.   :0.0000   Min.   :0.00000  
##  1st Qu.: 1525   1st Qu.:0.0   1st Qu.:0.0   1st Qu.:0.0000   1st Qu.:0.00000  
##  Median : 2788   Median :0.5   Median :0.5   Median :0.0000   Median :0.00000  
##  Mean   : 3330   Mean   :0.5   Mean   :0.5   Mean   :0.1056   Mean   :0.05185  
##  3rd Qu.: 4399   3rd Qu.:1.0   3rd Qu.:1.0   3rd Qu.:0.0000   3rd Qu.:0.00000  
##  Max.   :12110   Max.   :1.0   Max.   :1.0   Max.   :1.0000   Max.   :1.00000  
##                                                                                
##     ETHWHITE           AGE              S            EDUCPROF       
##  Min.   :0.0000   Min.   :37.00   Min.   : 7.00   Min.   :0.000000  
##  1st Qu.:1.0000   1st Qu.:39.00   1st Qu.:12.00   1st Qu.:0.000000  
##  Median :1.0000   Median :41.00   Median :13.00   Median :0.000000  
##  Mean   :0.8426   Mean   :40.92   Mean   :13.67   Mean   :0.007407  
##  3rd Qu.:1.0000   3rd Qu.:43.00   3rd Qu.:16.00   3rd Qu.:0.000000  
##  Max.   :1.0000   Max.   :45.00   Max.   :20.00   Max.   :1.000000  
##                                                                     
##     EDUCPHD            EDUCMAST           EDUCBA           EDUCAA       
##  Min.   :0.000000   Min.   :0.00000   Min.   :0.0000   Min.   :0.00000  
##  1st Qu.:0.000000   1st Qu.:0.00000   1st Qu.:0.0000   1st Qu.:0.00000  
##  Median :0.000000   Median :0.00000   Median :0.0000   Median :0.00000  
##  Mean   :0.001852   Mean   :0.05185   Mean   :0.1944   Mean   :0.08889  
##  3rd Qu.:0.000000   3rd Qu.:0.00000   3rd Qu.:0.0000   3rd Qu.:0.00000  
##  Max.   :1.000000   Max.   :1.00000   Max.   :1.0000   Max.   :1.00000  
##                                                                         
##     EDUCHSD           EDUCDO            SINGLE          MARRIED      
##  Min.   :0.0000   Min.   :0.00000   Min.   :0.0000   Min.   :0.0000  
##  1st Qu.:0.0000   1st Qu.:0.00000   1st Qu.:0.0000   1st Qu.:0.0000  
##  Median :1.0000   Median :0.00000   Median :0.0000   Median :1.0000  
##  Mean   :0.5481   Mean   :0.07778   Mean   :0.1481   Mean   :0.6574  
##  3rd Qu.:1.0000   3rd Qu.:0.00000   3rd Qu.:0.0000   3rd Qu.:1.0000  
##  Max.   :1.0000   Max.   :1.00000   Max.   :1.0000   Max.   :1.0000  
##                                                                      
##     DIVORCED          FAITHN            FAITHP            FAITHC       
##  Min.   :0.0000   Min.   :-3.0000   Min.   :-3.0000   Min.   :-3.0000  
##  1st Qu.:0.0000   1st Qu.: 0.0000   1st Qu.: 0.0000   1st Qu.: 0.0000  
##  Median :0.0000   Median : 0.0000   Median : 1.0000   Median : 0.0000  
##  Mean   :0.1944   Mean   : 0.0463   Mean   : 0.4963   Mean   : 0.3204  
##  3rd Qu.:0.0000   3rd Qu.: 0.0000   3rd Qu.: 1.0000   3rd Qu.: 1.0000  
##  Max.   :1.0000   Max.   : 1.0000   Max.   : 1.0000   Max.   : 1.0000  
##                                                                        
##      FAITHJ              FAITHO           ASVAB01      ASVAB02     
##  Min.   :-3.000000   Min.   :-3.0000   Min.   :22   Min.   :31.00  
##  1st Qu.: 0.000000   1st Qu.: 0.0000   1st Qu.:44   1st Qu.:42.00  
##  Median : 0.000000   Median : 0.0000   Median :50   Median :50.00  
##  Mean   :-0.003704   Mean   : 0.1111   Mean   :50   Mean   :50.48  
##  3rd Qu.: 0.000000   3rd Qu.: 0.0000   3rd Qu.:56   3rd Qu.:58.00  
##  Max.   : 1.000000   Max.   : 1.0000   Max.   :68   Max.   :66.00  
##                                                                    
##     ASVAB03         ASVAB04         ASVAB05         ASVAB06     
##  Min.   :24.00   Min.   :20.00   Min.   :20.00   Min.   :22.00  
##  1st Qu.:44.00   1st Qu.:47.00   1st Qu.:46.00   1st Qu.:45.00  
##  Median :52.00   Median :53.00   Median :52.00   Median :51.00  
##  Mean   :49.71   Mean   :50.36   Mean   :50.78   Mean   :50.22  
##  3rd Qu.:57.00   3rd Qu.:56.00   3rd Qu.:59.00   3rd Qu.:57.00  
##  Max.   :61.00   Max.   :62.00   Max.   :62.00   Max.   :72.00  
##                                                                 
##      ASVABC          HEIGHT         WEIGHT85        WEIGHT02         SM       
##  Min.   :25.46   Min.   :48.00   Min.   : 95.0   Min.   : 95   Min.   : 0.00  
##  1st Qu.:44.54   1st Qu.:64.00   1st Qu.:129.0   1st Qu.:150   1st Qu.:11.00  
##  Median :52.72   Median :67.00   Median :155.0   Median :180   Median :12.00  
##  Mean   :51.36   Mean   :67.67   Mean   :156.7   Mean   :184   Mean   :11.58  
##  3rd Qu.:58.72   3rd Qu.:71.00   3rd Qu.:178.0   3rd Qu.:210   3rd Qu.:12.00  
##  Max.   :66.08   Max.   :80.00   Max.   :322.0   Max.   :400   Max.   :20.00  
##                                                                               
##        SF           SIBLINGS         LIBRARY           POV78       
##  Min.   : 0.00   Min.   : 0.000   Min.   :0.0000   Min.   :0.0000  
##  1st Qu.:10.00   1st Qu.: 2.000   1st Qu.:1.0000   1st Qu.:0.0000  
##  Median :12.00   Median : 3.000   Median :1.0000   Median :0.0000  
##  Mean   :11.84   Mean   : 3.274   Mean   :0.7796   Mean   :0.1306  
##  3rd Qu.:14.00   3rd Qu.: 4.000   3rd Qu.:1.0000   3rd Qu.:0.0000  
##  Max.   :20.00   Max.   :13.000   Max.   :1.0000   Max.   :1.0000  
##                                                    NA's   :27      
##       EXP            EARNINGS          HOURS           TENURE        
##  Min.   : 1.154   Min.   :  2.13   Min.   :10.00   Min.   : 0.01923  
##  1st Qu.:14.596   1st Qu.: 10.76   1st Qu.:40.00   1st Qu.: 1.93750  
##  Median :17.510   Median : 16.00   Median :40.00   Median : 4.69231  
##  Mean   :16.900   Mean   : 19.64   Mean   :40.53   Mean   : 7.03397  
##  3rd Qu.:20.197   3rd Qu.: 23.16   3rd Qu.:45.00   3rd Qu.:10.98077  
##  Max.   :23.558   Max.   :120.19   Max.   :60.00   Max.   :24.94231  
##                                                                      
##     COLLBARG         CATGOV           CATPRI           CATSE        
##  Min.   :0.000   Min.   :0.0000   Min.   :0.0000   Min.   :0.00000  
##  1st Qu.:0.000   1st Qu.:0.0000   1st Qu.:1.0000   1st Qu.:0.00000  
##  Median :0.000   Median :0.0000   Median :1.0000   Median :0.00000  
##  Mean   :0.187   Mean   :0.2278   Mean   :0.7537   Mean   :0.01852  
##  3rd Qu.:0.000   3rd Qu.:0.0000   3rd Qu.:1.0000   3rd Qu.:0.00000  
##  Max.   :1.000   Max.   :1.0000   Max.   :1.0000   Max.   :1.00000  
##                                                                     
##      URBAN            REGNE            REGNC            REGW       
##  Min.   :0.0000   Min.   :0.0000   Min.   :0.000   Min.   :0.0000  
##  1st Qu.:0.0000   1st Qu.:0.0000   1st Qu.:0.000   1st Qu.:0.0000  
##  Median :1.0000   Median :0.0000   Median :0.000   Median :0.0000  
##  Mean   :0.7241   Mean   :0.1426   Mean   :0.337   Mean   :0.1574  
##  3rd Qu.:1.0000   3rd Qu.:0.0000   3rd Qu.:1.000   3rd Qu.:0.0000  
##  Max.   :2.0000   Max.   :1.0000   Max.   :1.000   Max.   :1.0000  
##                                                                    
##       REGS      
##  Min.   :0.000  
##  1st Qu.:0.000  
##  Median :0.000  
##  Mean   :0.363  
##  3rd Qu.:1.000  
##  Max.   :1.000  
## 
\end{verbatim}

\begin{Shaded}
\begin{Highlighting}[]
\FunctionTok{nrow}\NormalTok{(mydata)    }\CommentTok{\# gives the number of rows}
\end{Highlighting}
\end{Shaded}

\begin{verbatim}
## [1] 540
\end{verbatim}

\begin{Shaded}
\begin{Highlighting}[]
\FunctionTok{ncol}\NormalTok{(mydata)    }\CommentTok{\# gives the number of columns}
\end{Highlighting}
\end{Shaded}

\begin{verbatim}
## [1] 51
\end{verbatim}

The object mydata is a data.frame, i.e.~a dataset with named columns. We
can perform operations by rows and by columns with function apply.

\begin{Shaded}
\begin{Highlighting}[]
\CommentTok{\# this will apply function class to each column of the dataset}
\FunctionTok{apply}\NormalTok{(mydata,     }\CommentTok{\# the dataset to work on}
      \AttributeTok{MARGIN =} \DecValTok{2}\NormalTok{, }\CommentTok{\# 1 to work by rows and 2 by columns}
      \AttributeTok{FUN =}\NormalTok{ mean) }\CommentTok{\# the function to apply}
\end{Highlighting}
\end{Shaded}

\begin{verbatim}
##            ID        FEMALE          MALE      ETHBLACK       ETHHISP 
##  3.330191e+03  5.000000e-01  5.000000e-01  1.055556e-01  5.185185e-02 
##      ETHWHITE           AGE             S      EDUCPROF       EDUCPHD 
##  8.425926e-01  4.091852e+01  1.367222e+01  7.407407e-03  1.851852e-03 
##      EDUCMAST        EDUCBA        EDUCAA       EDUCHSD        EDUCDO 
##  5.185185e-02  1.944444e-01  8.888889e-02  5.481481e-01  7.777778e-02 
##        SINGLE       MARRIED      DIVORCED        FAITHN        FAITHP 
##  1.481481e-01  6.574074e-01  1.944444e-01  4.629630e-02  4.962963e-01 
##        FAITHC        FAITHJ        FAITHO       ASVAB01       ASVAB02 
##  3.203704e-01 -3.703704e-03  1.111111e-01  4.999630e+01  5.047593e+01 
##       ASVAB03       ASVAB04       ASVAB05       ASVAB06        ASVABC 
##  4.970926e+01  5.035556e+01  5.077963e+01  5.022222e+01  5.136271e+01 
##        HEIGHT      WEIGHT85      WEIGHT02            SM            SF 
##  6.767407e+01  1.567481e+02  1.840130e+02  1.157963e+01  1.183704e+01 
##      SIBLINGS       LIBRARY         POV78           EXP      EARNINGS 
##  3.274074e+00  7.796296e-01            NA  1.690036e+01  1.963622e+01 
##         HOURS        TENURE      COLLBARG        CATGOV        CATPRI 
##  4.053148e+01  7.033974e+00  1.870370e-01  2.277778e-01  7.537037e-01 
##         CATSE         URBAN         REGNE         REGNC          REGW 
##  1.851852e-02  7.240741e-01  1.425926e-01  3.370370e-01  1.574074e-01 
##          REGS 
##  3.629630e-01
\end{verbatim}

\begin{Shaded}
\begin{Highlighting}[]
\FunctionTok{rm}\NormalTok{(gdp\_data) }\CommentTok{\# remove R object from environment}
\end{Highlighting}
\end{Shaded}

\hypertarget{tidyverse}{%
\subsection{Tidyverse}\label{tidyverse}}

Tidyverse is a nice R package combining several packages for
datamanagement and plotting, including \textbf{dplyr} \textbf{tidyr} and
\textbf{ggplot2}. This package allows us to easily manipulate data.

\begin{Shaded}
\begin{Highlighting}[]
\ControlFlowTok{if}\NormalTok{(}\SpecialCharTok{!}\FunctionTok{require}\NormalTok{(tidyverse))\{}\FunctionTok{install.packages}\NormalTok{(}\StringTok{"tidyverse"}\NormalTok{)\}}
\FunctionTok{library}\NormalTok{(tidyverse)          }\CommentTok{\# Package for data manipulation}

\CommentTok{\# the following code gives same result as }
\CommentTok{\# mean(mydata$EARNINGS)}
\NormalTok{mydata }\SpecialCharTok{\%\textgreater{}\%} 
  \FunctionTok{summarize}\NormalTok{(}\FunctionTok{mean}\NormalTok{(EARNINGS))}
\end{Highlighting}
\end{Shaded}

\begin{verbatim}
##   mean(EARNINGS)
## 1       19.63622
\end{verbatim}

Create new varables with mutate.

\begin{Shaded}
\begin{Highlighting}[]
\CommentTok{\# create and add a new variable}
\NormalTok{mydata }\SpecialCharTok{\%\textgreater{}\%} 
  \FunctionTok{mutate}\NormalTok{(}\AttributeTok{EDUCUNI =}\NormalTok{ EDUCBA }\SpecialCharTok{+}\NormalTok{ EDUCMAST) }\SpecialCharTok{\%\textgreater{}\%}
  \FunctionTok{head}\NormalTok{(}\DecValTok{3}\NormalTok{)}
\end{Highlighting}
\end{Shaded}

\begin{verbatim}
##     ID FEMALE MALE ETHBLACK ETHHISP ETHWHITE AGE  S EDUCPROF EDUCPHD EDUCMAST
## 1 5531      0    1        0       0        1  45 12        0       0        0
## 2 2658      0    1        0       1        0  40 12        0       0        0
## 3 5365      0    1        0       0        1  38 15        0       0        0
##   EDUCBA EDUCAA EDUCHSD EDUCDO SINGLE MARRIED DIVORCED FAITHN FAITHP FAITHC
## 1      0      0       1      0      0       1        0      0      1      0
## 2      0      0       0      1      0       0        1      1      0      0
## 3      0      1       0      0      0       0        1      0      0      0
##   FAITHJ FAITHO ASVAB01 ASVAB02 ASVAB03 ASVAB04 ASVAB05 ASVAB06   ASVABC HEIGHT
## 1      0      0      58      64      52      56      54      56 60.89985     67
## 2      0      0      32      39      29      29      27      22 33.63790     67
## 3      0      1      42      40      37      38      42      45 38.81767     69
##   WEIGHT85 WEIGHT02 SM SF SIBLINGS LIBRARY POV78       EXP EARNINGS HOURS
## 1      160      200  8  8       11       0     0 22.384615    53.41    45
## 2      185      205  5  5        3       0     1  8.903846     8.00    40
## 3      135      185 11 12        3       1     0 13.250000    24.00    40
##     TENURE COLLBARG CATGOV CATPRI CATSE URBAN REGNE REGNC REGW REGS EDUCUNI
## 1 2.750000        0      0      1     0     0     0     0    0    1       0
## 2 2.384615        0      0      1     0     0     0     0    0    1       0
## 3 5.750000        1      0      1     0     1     0     1    0    0       0
\end{verbatim}

Perform operations by clusters.

\begin{Shaded}
\begin{Highlighting}[]
\CommentTok{\# mutate/summarize by groups}
\NormalTok{mydata }\SpecialCharTok{\%\textgreater{}\%} 
  \FunctionTok{group\_by}\NormalTok{(FEMALE) }\SpecialCharTok{\%\textgreater{}\%}
  \FunctionTok{summarize}\NormalTok{(}\FunctionTok{mean}\NormalTok{(EXP))}
\end{Highlighting}
\end{Shaded}

\begin{verbatim}
## # A tibble: 2 x 2
##   FEMALE `mean(EXP)`
##    <dbl>       <dbl>
## 1      0        17.9
## 2      1        15.9
\end{verbatim}

\begin{Shaded}
\begin{Highlighting}[]
\NormalTok{mydata }\SpecialCharTok{\%\textgreater{}\%} 
  \FunctionTok{group\_by}\NormalTok{(MARRIED) }\SpecialCharTok{\%\textgreater{}\%}
  \FunctionTok{mutate}\NormalTok{(}\AttributeTok{DHOURS =}\NormalTok{ HOURS }\SpecialCharTok{{-}} \FunctionTok{mean}\NormalTok{(HOURS)) }\SpecialCharTok{\%\textgreater{}\%}
  \FunctionTok{head}\NormalTok{()}
\end{Highlighting}
\end{Shaded}

\begin{verbatim}
## # A tibble: 6 x 52
## # Groups:   MARRIED [2]
##      ID FEMALE  MALE ETHBLACK ETHHISP ETHWHITE   AGE     S EDUCPROF EDUCPHD
##   <dbl>  <dbl> <dbl>    <dbl>   <dbl>    <dbl> <dbl> <dbl>    <dbl>   <dbl>
## 1  5531      0     1        0       0        1    45    12        0       0
## 2  2658      0     1        0       1        0    40    12        0       0
## 3  5365      0     1        0       0        1    38    15        0       0
## 4  4468      0     1        0       0        1    43    13        0       0
## 5  3142      0     1        0       0        1    38    18        0       0
## 6  2170      0     1        1       0        0    39    16        0       0
## # i 42 more variables: EDUCMAST <dbl>, EDUCBA <dbl>, EDUCAA <dbl>,
## #   EDUCHSD <dbl>, EDUCDO <dbl>, SINGLE <dbl>, MARRIED <dbl>, DIVORCED <dbl>,
## #   FAITHN <dbl>, FAITHP <dbl>, FAITHC <dbl>, FAITHJ <dbl>, FAITHO <dbl>,
## #   ASVAB01 <dbl>, ASVAB02 <dbl>, ASVAB03 <dbl>, ASVAB04 <dbl>, ASVAB05 <dbl>,
## #   ASVAB06 <dbl>, ASVABC <dbl>, HEIGHT <dbl>, WEIGHT85 <dbl>, WEIGHT02 <dbl>,
## #   SM <dbl>, SF <dbl>, SIBLINGS <dbl>, LIBRARY <dbl>, POV78 <dbl>, EXP <dbl>,
## #   EARNINGS <dbl>, HOURS <dbl>, TENURE <dbl>, COLLBARG <dbl>, ...
\end{verbatim}

Perform more operations at the same time

\begin{Shaded}
\begin{Highlighting}[]
\CommentTok{\# you can also \textquotesingle{}filter\textquotesingle{} rows and \textquotesingle{}select\textquotesingle{} columns}
\NormalTok{mydata }\SpecialCharTok{\%\textgreater{}\%}
  \FunctionTok{mutate}\NormalTok{(}\AttributeTok{AGE\_D =} \FunctionTok{ifelse}\NormalTok{(AGE }\SpecialCharTok{\textless{}=} \DecValTok{40}\NormalTok{, }\StringTok{"below 40"}\NormalTok{, }\StringTok{"above 40"}\NormalTok{)) }\SpecialCharTok{\%\textgreater{}\%}
  \FunctionTok{group\_by}\NormalTok{(AGE\_D) }\SpecialCharTok{\%\textgreater{}\%}
  \FunctionTok{summarise}\NormalTok{(}\FunctionTok{mean}\NormalTok{(EARNINGS), }\FunctionTok{mean}\NormalTok{(HOURS))}
\end{Highlighting}
\end{Shaded}

\begin{verbatim}
## # A tibble: 2 x 3
##   AGE_D    `mean(EARNINGS)` `mean(HOURS)`
##   <chr>               <dbl>         <dbl>
## 1 above 40             19.9          40.3
## 2 below 40             19.3          40.8
\end{verbatim}

Pick only columns and observations we are interested into.

\begin{Shaded}
\begin{Highlighting}[]
\CommentTok{\# you can also \textquotesingle{}filter\textquotesingle{} rows and \textquotesingle{}select\textquotesingle{} columns}
\NormalTok{mydata }\SpecialCharTok{\%\textgreater{}\%} 
  \FunctionTok{filter}\NormalTok{(AGE }\SpecialCharTok{\textgreater{}} \DecValTok{40}\NormalTok{) }\SpecialCharTok{\%\textgreater{}\%}
  \FunctionTok{select}\NormalTok{(EARNINGS, EXP, FEMALE) }\SpecialCharTok{\%\textgreater{}\%}
  \FunctionTok{head}\NormalTok{()}
\end{Highlighting}
\end{Shaded}

\begin{verbatim}
##   EARNINGS      EXP FEMALE
## 1    53.41 22.38461      0
## 2    29.50 18.25000      0
## 3    11.78 22.03846      0
## 4    27.30 16.38461      0
## 5    55.05 17.30769      0
## 6    12.30 22.40385      0
\end{verbatim}

In the above example we are using a pipe ``\%\textgreater\%''. This
comes from the package magrittr and is compatible with all tidyverse
packages.

Here we complete the same but ussing the classic R approach

\begin{Shaded}
\begin{Highlighting}[]
\CommentTok{\# classic R}
\NormalTok{classic }\OtherTok{\textless{}{-}} \FunctionTok{filter}\NormalTok{(mydata, AGE }\SpecialCharTok{\textgreater{}} \DecValTok{40}\NormalTok{)}
\NormalTok{classic1 }\OtherTok{\textless{}{-}} \FunctionTok{select}\NormalTok{(classic,}\FunctionTok{c}\NormalTok{(EARNINGS, EXP, FEMALE))}
\NormalTok{classic2 }\OtherTok{\textless{}{-}} \FunctionTok{head}\NormalTok{(classic1)}
\FunctionTok{print}\NormalTok{(classic2)}
\end{Highlighting}
\end{Shaded}

\begin{verbatim}
##   EARNINGS      EXP FEMALE
## 1    53.41 22.38461      0
## 2    29.50 18.25000      0
## 3    11.78 22.03846      0
## 4    27.30 16.38461      0
## 5    55.05 17.30769      0
## 6    12.30 22.40385      0
\end{verbatim}

Now, let's put that in practice! Can you write some code to compare the
average male and female earnings for above 40 year olds with a masters

\begin{Shaded}
\begin{Highlighting}[]
\NormalTok{avg\_earnings }\OtherTok{\textless{}{-}}\NormalTok{ mydata }\SpecialCharTok{\%\textgreater{}\%}
  \FunctionTok{filter}\NormalTok{(AGE }\SpecialCharTok{\textgreater{}} \DecValTok{40}\NormalTok{, EDUCMAST }\SpecialCharTok{==} \DecValTok{1}\NormalTok{) }\SpecialCharTok{\%\textgreater{}\%}
  \FunctionTok{group\_by}\NormalTok{(MALE) }\SpecialCharTok{\%\textgreater{}\%}
  \FunctionTok{summarize}\NormalTok{(}\AttributeTok{AverageEarnings =} \FunctionTok{mean}\NormalTok{(EARNINGS, }\AttributeTok{na.rm =} \ConstantTok{TRUE}\NormalTok{))}

\FunctionTok{head}\NormalTok{(avg\_earnings)}
\end{Highlighting}
\end{Shaded}

\begin{verbatim}
## # A tibble: 2 x 2
##    MALE AverageEarnings
##   <dbl>           <dbl>
## 1     0            23.7
## 2     1            30.3
\end{verbatim}

\begin{Shaded}
\begin{Highlighting}[]
\CommentTok{\# 0 represents women}
\CommentTok{\# 1 represents men}
\end{Highlighting}
\end{Shaded}

\hypertarget{univariate-analysis}{%
\section{Univariate analysis}\label{univariate-analysis}}

We can also easily plot our data. We can use histograms for numerical
variables.

\begin{Shaded}
\begin{Highlighting}[]
\NormalTok{mydata }\SpecialCharTok{\%\textgreater{}\%} 
  \FunctionTok{ggplot}\NormalTok{(}\FunctionTok{aes}\NormalTok{(EARNINGS)) }\SpecialCharTok{+}
    \FunctionTok{geom\_histogram}\NormalTok{(}\AttributeTok{bins =} \DecValTok{35}\NormalTok{, }
                   \AttributeTok{color =} \StringTok{"black"}\NormalTok{, }
                   \AttributeTok{fill =} \StringTok{"white"}\NormalTok{) }\SpecialCharTok{+}
    \FunctionTok{ylab}\NormalTok{(}\StringTok{""}\NormalTok{) }\SpecialCharTok{+} 
    \FunctionTok{xlab}\NormalTok{(}\StringTok{""}\NormalTok{) }\SpecialCharTok{+}
    \FunctionTok{ggtitle}\NormalTok{(}\StringTok{"Histograms of Earnings (Frequency)"}\NormalTok{)  }
\end{Highlighting}
\end{Shaded}

\includegraphics{EF_Tutorial_1_files/figure-latex/unnamed-chunk-19-1.pdf}

We can add a density line, but we need to show histograms as
probabilities.

\begin{Shaded}
\begin{Highlighting}[]
\NormalTok{mydata }\SpecialCharTok{\%\textgreater{}\%} 
  \FunctionTok{ggplot}\NormalTok{(}\FunctionTok{aes}\NormalTok{(EARNINGS)) }\SpecialCharTok{+}
    \FunctionTok{geom\_histogram}\NormalTok{(}\FunctionTok{aes}\NormalTok{(}\AttributeTok{y =} \FunctionTok{after\_stat}\NormalTok{(density)),}
                   \AttributeTok{bins =} \DecValTok{100}\NormalTok{, }
                   \AttributeTok{color =} \StringTok{"grey30"}\NormalTok{, }
                   \AttributeTok{fill =} \StringTok{"white"}\NormalTok{) }\SpecialCharTok{+}
        \FunctionTok{geom\_density}\NormalTok{(}\AttributeTok{alpha =}\NormalTok{ .}\DecValTok{2}\NormalTok{, }
                     \AttributeTok{fill =} \StringTok{"antiquewhite3"}\NormalTok{) }\SpecialCharTok{+}
    \FunctionTok{ylab}\NormalTok{(}\StringTok{""}\NormalTok{) }\SpecialCharTok{+} 
    \FunctionTok{xlab}\NormalTok{(}\StringTok{""}\NormalTok{) }\SpecialCharTok{+}
    \FunctionTok{ggtitle}\NormalTok{(}\StringTok{"Histograms of Earnings (Density)"}\NormalTok{)  }
\end{Highlighting}
\end{Shaded}

\includegraphics{EF_Tutorial_1_files/figure-latex/unnamed-chunk-20-1.pdf}

For qualitative analysis you can try for yourself geom\_bar() and
geom\_count().

\hypertarget{univariate-regression}{%
\subsection{Univariate regression}\label{univariate-regression}}

\begin{Shaded}
\begin{Highlighting}[]
\NormalTok{mod0 }\OtherTok{\textless{}{-}}\NormalTok{ mydata }\SpecialCharTok{\%\textgreater{}\%}
  \FunctionTok{lm}\NormalTok{(}\FunctionTok{log}\NormalTok{(EARNINGS) }\SpecialCharTok{\textasciitilde{}}\NormalTok{ EXP, }\AttributeTok{data =}\NormalTok{ .)}
  
\NormalTok{mod0 }\SpecialCharTok{\%\textgreater{}\%} \FunctionTok{summary}\NormalTok{()}
\end{Highlighting}
\end{Shaded}

\begin{verbatim}
## 
## Call:
## lm(formula = log(EARNINGS) ~ EXP, data = .)
## 
## Residuals:
##      Min       1Q   Median       3Q      Max 
## -2.00515 -0.41689 -0.03241  0.35866  2.02727 
## 
## Coefficients:
##             Estimate Std. Error t value Pr(>|t|)    
## (Intercept) 2.449410   0.098823  24.786  < 2e-16 ***
## EXP         0.020271   0.005656   3.584  0.00037 ***
## ---
## Signif. codes:  0 '***' 0.001 '**' 0.01 '*' 0.05 '.' 0.1 ' ' 1
## 
## Residual standard error: 0.5822 on 538 degrees of freedom
## Multiple R-squared:  0.02332,    Adjusted R-squared:  0.0215 
## F-statistic: 12.84 on 1 and 538 DF,  p-value: 0.0003695
\end{verbatim}

\hypertarget{multivariate-analysis}{%
\section{Multivariate analysis}\label{multivariate-analysis}}

We can now start with multiple linear regression model.

\begin{Shaded}
\begin{Highlighting}[]
\NormalTok{mod1 }\OtherTok{\textless{}{-}}\NormalTok{ mydata }\SpecialCharTok{\%\textgreater{}\%}
  \FunctionTok{lm}\NormalTok{(}\FunctionTok{log}\NormalTok{(EARNINGS) }\SpecialCharTok{\textasciitilde{}}\NormalTok{ AGE }\SpecialCharTok{+}\NormalTok{ EXP }\SpecialCharTok{+} 
\NormalTok{       EDUCPROF }\SpecialCharTok{+}\NormalTok{ EDUCPHD }\SpecialCharTok{+} 
\NormalTok{       EDUCMAST }\SpecialCharTok{+}\NormalTok{ EDUCBA }\SpecialCharTok{+} 
\NormalTok{       EDUCAA }\SpecialCharTok{+}\NormalTok{ EDUCHSD }\SpecialCharTok{+} 
\NormalTok{       EDUCDO, }\AttributeTok{data =}\NormalTok{ .)}
  
\NormalTok{mod1 }\SpecialCharTok{\%\textgreater{}\%} \FunctionTok{summary}\NormalTok{()}
\end{Highlighting}
\end{Shaded}

\begin{verbatim}
## 
## Call:
## lm(formula = log(EARNINGS) ~ AGE + EXP + EDUCPROF + EDUCPHD + 
##     EDUCMAST + EDUCBA + EDUCAA + EDUCHSD + EDUCDO, data = .)
## 
## Residuals:
##      Min       1Q   Median       3Q      Max 
## -1.77593 -0.32854 -0.03132  0.30212  2.01786 
## 
## Coefficients:
##              Estimate Std. Error t value Pr(>|t|)    
## (Intercept)  3.671919   0.411639   8.920  < 2e-16 ***
## AGE         -0.030217   0.009872  -3.061 0.002319 ** 
## EXP          0.036813   0.005303   6.942 1.13e-11 ***
## EDUCPROF     1.322317   0.281213   4.702 3.29e-06 ***
## EDUCPHD      0.135952   0.518411   0.262 0.793232    
## EDUCMAST     0.244102   0.158043   1.545 0.123057    
## EDUCBA       0.123178   0.134919   0.913 0.361672    
## EDUCAA      -0.247465   0.145376  -1.702 0.089296 .  
## EDUCHSD     -0.437111   0.129628  -3.372 0.000801 ***
## EDUCDO      -0.651896   0.148039  -4.404 1.29e-05 ***
## ---
## Signif. codes:  0 '***' 0.001 '**' 0.01 '*' 0.05 '.' 0.1 ' ' 1
## 
## Residual standard error: 0.5024 on 530 degrees of freedom
## Multiple R-squared:  0.2835, Adjusted R-squared:  0.2714 
## F-statistic: 23.31 on 9 and 530 DF,  p-value: < 2.2e-16
\end{verbatim}

\end{document}
